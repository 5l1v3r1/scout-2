\let\clearforchapter\par % cheating, but saves some space


\chapter{Visualization Constructs}

A key feature of the Scout language is the incorporation of visualization and
rendering operations directly within the syntax and semantics of the language 
as first-class constructs.  
The \texttt{window} is also a Scout construct.  Windows are constructed so that images can be displayed on them.   Currently 2D rendering is supported.
Listing~\ref{renderall1} shows a simple example.

\par\bigskip
\begin{lstlisting}[float=h,label=renderall1,
	caption={A \texttt{renderall} loop construct.}]
	// Example visualization construct for rendering mesh cells. 
	uniform mesh meshType {
		cells: float a, b;
	};

  meshType myMesh[16];
  window win[1024,1024] {

	renderall cells c in myMesh to win
  {
			color = rgb(1.0, 0.0, 0.0);
		else
			color = rgb(0.0, 0.0, 1.0);
	}
\end{lstlisting}
\par\bigskip\noindent

Note that when meshes are passed as parameters to functions, the \texttt{renderall} construct works via an indirection to a mesh pointer as well.
Listing~\ref{renderall-meshptr} shows a simple example.

\par\bigskip
\begin{lstlisting}[float=h,label=renderall-meshptr,
	caption={A \texttt{renderall} loop construct on mesh parameter.}]
void MyFunc(MyMesh *m) { 
  window mywin[512,512]; 
  
  ...

  renderall vertices v in *m to mywin{
    color = hsva(b/16.0f*360.0f, 1.0, 1.0, 1.0);
  }
\end{lstlisting}
\par\bigskip\noindent


%In the case where the mesh is three-dimensional, the body of the \texttt{renderall}
%is essentially the transfer function for volume rendering.

%% TBD
%The \texttt{window} is also a Scout construct.  Windows are constructed so that images can
%be displayed on them given a camera.  Windows can have multiple viewports within them.  
%A \texttt{viewport} has a width and height and offset from lower left corner.
%Windows and viewports are constructed as shown in Listing~\ref{window1}. 

%\par\bigskip
%\begin{lstlisting}[float=h,label=window1,
%	caption={How to declare a \texttt{window}.}]
%  window win[1024,1024] {
%    background  = hsv(0.1, 0.2, 0.3);
%    save_frames = true;
%    filename    = "heat2d-####.png";
%  };
%
%  viewport aviewport[512, 512] {
%    x_position = 10.0;  // offset from lower left corner
%    y_position = 10.0;
%  };
%
%  awindow.add(aviewport);
%
%  awindow.display();
%
%  // assume a camera and a three-dimensional mesh have been defined
%
%  renderall cells c of amesh with acamera onto aviewport {
%  // ....
%  }
%
%  awindow.delete();  // deletes viewports too
%\end{lstlisting} 
%\par\bigskip\noindent
%%

