\chapter*{Preface}

\textsf{scout \textsl{verb} [intrans.] -- \textit{to explore or examine so as to gather
information.}}
\vskip 0.25in

\section*{Welcome}
The goal of this manual is to provide you with a quick reference to the syntax, 
semantics and features of the experimental Scout language.  We assume that the
reader is an experienced programmer with a basic understanding of parallel 
computing.  Although we do our best to keep the language documentation up to 
date with the feature set supported by the open-source versions of Scout, readers
are encouraged to look over the release notes provided with each version for 
important details.

\section*{About Source Code Listings}

The source code listings in this manual uses both font changes and \textit{syntax aware} 
coloring that helps the reader to identify parts of the language.  Table~\ref{src_listing_table}
provides a key to the font and colors using in the listings. 

\begin{table}
  \begin{center}
  \begin{tabular}{| l | l | l |}
    \hline
 	\textsc{Type}      & \textsc{Color}                           & \textsc{Font}     \\ \hline 
	\textsc{keywords}  & {\texttt{\color{keywordcolor}keyword}}   & \texttt{keyword}  \\ \hline 
    \textsc{built-ins} & {\texttt{\color{builtincolor}built-ins}} & \texttt{built-in} \\ \hline 
    \textsc{comments}  & {\textit{\color{commentcolor}comments}}  & \textit{comments} \\ \hline 
    \textsc{strings}   & {\textit{\color{stringcolor}"string"}}   & \textit{"string"} \\ \hline 
  \end{tabular}
  \end{center}
  \caption{Source code listing colors and fonts.}
  \label{src_listing_table} 
\end{table}


\section*{Support}
If you have questions, or encounter problems, please feel free to send an email
to the Scout support team via email: {\color{scoutblue}scout-support@lanl.gov}.

\section*{Open Source Effort}
This version of Scout is an open-source software effort established by Los Alamos
National Laboratory's Applied Computer Science Group. The source code is available here: 
\begin{center} 
	{\color{scoutblue}\url{https://github.com/losalamos/scout}}.
\end{center}

If you are interested in learning more about the LANL team working on Scout, please visit our LANL web site here:
\begin{center} 
	{\color{scoutblue}\url{http://progmodels.lanl.gov/scout}}.
\end{center}






